\documentclass[12pt]{article}
\usepackage[svgnames,x11names,table]{xcolor}
\usepackage{hyperref}
\usepackage{graphicx}
\usepackage{parskip}
\usepackage{float}
\usepackage{amsmath}
\usepackage{amssymb}
\usepackage{enumitem}
\usepackage[thicklines]{cancel}

\hypersetup{
    colorlinks,
    citecolor=black,
    filecolor=black,
    linkcolor=RoyalBlue4,
    urlcolor=RoyalBlue4,
}

\title{PEU 438 Assignment 1}
\author{Mohamed Hussien El-Deeb (201900052)}
\date{}

\begin{document}

\maketitle
\tableofcontents

\newcommand{\Lagr}{\mathcal{L}}

\newpage

\section{Problem 1}

\subsection{Blackbody Radiation Energy Inside the Eye}

Radius of the Eye $(r)$:

\[
    r = 1.5\, \text{cm} = 0.015\, \text{m}
\]

Volume of the Eye $(V_{eye})$:

\[
    V_{\text{eye}} = \frac{4}{3} \pi r^3
    = \frac{4}{3} \pi {(0.015\, \text{m})}^3 = 1.4137 \times 10^{-5}\, \text{m}^3
\]

Energy Density of Blackbody Radiation (u):

\[
    u = a T^4
\]

Where
\( a = 7.5657 \times 10^{-16}\, \text{J}\cdot\text{m}^{-3}\cdot\text{K}^{-4} \)
is the radiation constant,
and \( T = 37^\circ\text{C} = 310.15\, \text{K} \).

\[
    T^4 = {(310.15\, \text{K})}^4 = 9.254 \times 10^9\, \text{K}^4
\]

Then,

\[
    u =
    (7.5657 \times 10^{-16}\, \text{J}\cdot\text{m}^{-3}\cdot\text{K}^{-4})(9.254 \times 10^9\, \text{K}^4)
    = 7.0013 \times 10^{-6}\, \text{J/m}^3
\]

Total Energy Inside the Eye $(E_{eye})$:

\[
    E_{\text{eye}}
    = u \times V_{\text{eye}}
    = (7.0013 \times 10^{-6}\, \text{J/m}^3)(1.4137 \times 10^{-5}\, \text{m}^3)
    = 9.9 \times 10^{-11}\, \text{J}
\]

\subsection{Energy Entering the Eye from Light Bulb}

Intensity at 1 Meter (I):

\[
    I
    = \frac{\text{Power}}{4\pi r^2}
    = \frac{100\, \text{W}}{4\pi {(1\, \text{m})}^2}
    = 7.9577\, \text{W/m}^2
\]

Area of the Pupil $(A_{pupil})$:

\[
    A_{\text{pupil}}
    = 0.1\, \text{cm}^2
    = 1 \times 10^{-5}\, \text{m}^2
\]

Power Entering the Eye $(P_{eye})$:

\[
    P_{\text{eye}}
    = I \times A_{\text{pupil}}
    = (7.9577\, \text{W/m}^2)(1 \times 10^{-5}\, \text{m}^2)
    = 7.9577 \times 10^{-5}\, \text{W}
\]

Energy Entering the Eye in 1 Second $(E_{in})$:

\[
    E_{\text{in}}
    = P_{\text{eye}} \times t
    = (7.9577 \times 10^{-5}\, \text{W})(1\, \text{s})
    = 7.9577 \times 10^{-5}\, \text{J}
\]

\subsection{Comparison}

\[
    \frac{E_{\text{in}}}{E_{\text{eye}}}
    = \frac{7.9577 \times 10^{-5}\, \text{J}}{9.9 \times 10^{-11}\, \text{J}} \approx 8 \times 10^5
\]

\subsection{Answer}

It is dark when we close our eyes because the energy of the blackbody photons inside the eye is
below the threshold needed to stimulate the photoreceptors.

\newpage

\section{Problem 2}

\subsection{General Solution}

\[
    I_\nu = I_\nu(\tau_\nu=0) e^{-\tau_{\lambda,0}} + \int_{0}^{\tau_{\lambda,0}} S_\nu(\tau_\nu) e^{-(\tau_{\lambda,0} - \tau_\nu)} \, d\tau_\nu
\]

\[
    I_\nu(\tau_\nu=0) = 0,\quad S_\nu(\tau_\nu) = S_\nu
\]

\[
    I_\nu = S_\nu e^{-\tau_{\lambda,0}} \int_{0}^{\tau_{\lambda,0}} e^{\tau_\nu} \, d\tau_\nu
\]

\[
    I_\nu = S_\nu (1 - e^{-\tau_{\lambda,0}})
\]

\subsection{Small Optical Depth}

\[
    e^{-\tau_{\lambda,0}} \approx 1 - \tau_{\lambda,0}
\]

\[
    I_\nu = S_\nu \tau_{\lambda,0}
\]

\[
    S_\nu = \frac{j_\nu}{\alpha_\nu},\quad \tau_{\lambda,0} = \alpha_\nu L
\]

\[
    I_\nu = j_\nu L
\]

Where $j_\nu$ is the emission function of a specific wavelength.

\subsection{Large Optical Depth}

\[
    I_\nu = S_\nu
\]

\[
    I_\nu = B_\lambda(T) = \frac{2 h c^2}{\lambda^5} \cdot \frac{1}{e^{\frac{h c}{\lambda k_B T}} - 1}
\]

\newpage

\section{Problem 3}

\subsection{General Solution}

\[
    I_\nu = I_\nu(\tau_\nu=0) e^{-\tau_{\lambda,0}} + \int_{0}^{\tau_{\lambda,0}} S_\nu(\tau_\nu) e^{-(\tau_{\lambda,0} - \tau_\nu)} \, d\tau_\nu
\]

\[
    I_\nu(\tau_\nu=0) = I_{\lambda,0},\quad S_\nu(\tau_\nu) = S_\nu
\]

\[
    I_\nu = I_{\lambda,0} e^{-\tau_{\lambda,0}} + S_\nu e^{-\tau_{\lambda,0}} \int_{0}^{\tau_{\lambda,0}} e^{\tau_\nu} \, d\tau_\nu
\]

\[
    I_\nu = I_{\lambda,0} e^{-\tau_{\lambda,0}} + S_\nu (1 - e^{-\tau_{\lambda,0}})
\]

\subsection{Small Optical Depth}

\[
    e^{-\tau_{\lambda,0}} \approx 1 - \tau_{\lambda,0}
\]

\[
    I_\nu = (1 - \tau_{\lambda,0})I_{\lambda,0} + \tau_{\lambda,0}S_\nu = I_{\lambda,0} + (S_\nu - I_{\lambda,0})\tau_{\lambda,0}
\]

\[
    S_\nu = \frac{j_\nu}{\alpha_\nu},\quad \tau_{\lambda,0} = \alpha_\nu L
\]

\[
    I_\nu = I_{\lambda,0} (1 - \alpha_\nu L) + j_\nu L = I_{\lambda,0} + (j_\nu - \alpha_\nu I_{\lambda,0})L
\]

Where $j_\nu$ is the emission function of a specific wavelength and $a_\nu$ is the absorption coefficient for a specific wavelength.

\subsection{Large Optical Depth}

\[
    I_\nu = S_\nu
\]

\[
    I_\nu = B_\lambda(T) = \frac{2 h c^2}{\lambda^5} \cdot \frac{1}{e^{\frac{h c}{\lambda k_B T}} - 1}
\]

\newpage

\section{Problem 4}

\subsection{Brightness Temperature and Energy Regime}

\subsubsection{Brightness Temperature}

\bigskip

\textbf{Convert Angular Diameter to Radians}

{\footnotesize
    \[
        \theta = 4.3' \times \left( \dfrac{1\deg}{60'} \right) \times \left( \dfrac{\pi\ \text{rad}}{180\deg} \right) = 4.3 \times \dfrac{\pi}{10800} \approx 1.2507 \times 10^{-3}\ \text{rad}
    \]
}

\textbf{Calculate the Solid Angle $\Omega$}

{\footnotesize
    \[
        \Omega = \pi {\left( \dfrac{\theta}{2} \right)}^2 = \pi {\left( \dfrac{1.2507 \times 10^{-3}\ \text{rad}}{2} \right)}^2 = \pi {\left( 6.2535 \times 10^{-4}\ \text{rad} \right)}^2 \approx 1.2277 \times 10^{-6}\ \text{sr}
    \]
}

\textbf{Convert Flux Density to SI Units}

{\tiny
    \[
        F_{100} = \left(1.6 \times 10^{-19} \frac{\text{erg}}{\text{cm}^2\ \text{s}\ \text{Hz}}\right) \times \left( \dfrac{1 \times 10^{-7}\text{J}}{1\ \text{erg}} \right) \times \left( \dfrac{1}{{(1 \times 10^{-2}\ \text{m})}^2} \right) = 1.6 \times 10^{-19} \times 1 \times 10^{-7} \times 1 \times 10^{4} \frac{\text{W}}{\text{m}^2\ \text{Hz}}
    \]
}

\[
    F_{100} = 1.6 \times 10^{-22} \frac{\text{W}}{\text{m}^2\ \text{Hz}}
\]

\textbf{Find the Wavelength $\lambda$}

\[
    \lambda = \dfrac{c}{\nu} = \dfrac{3.0 \times 10^8\text{m s}^{-1}}{1.0 \times 10^8\text{Hz}} = 3.0\text{m}
\]

\textbf{Brightness Temperature}

\[
    T_b = \dfrac{F_\nu \lambda^2}{2 k \Omega}
\]

Where:
\begin{itemize}
    \item $F_\nu$ is the flux density at frequency $\nu$
    \item $\lambda$ is the wavelength
    \item $k$ is Boltzmann's constant ($1.38 \times 10^{-23}\text{J K}^{-1}$)
\end{itemize}

Substitute the values:

\[
    T_b = \dfrac{(1.6 \times 10^{-22}\text{W m}^{-2}\text{Hz}^{-1}) \times {(3.0\text{m})}^2}{2 \times (1.38 \times 10^{-23}\text{J K}^{-1}) \times (1.2277 \times 10^{-6}\text{sr})} = 4.247 \times 10^{7}\text{K}
\]

\subsubsection{Energy Regime}

Compare $h \nu$ and $k T_b$:

\[
    h \nu = (6.626 \times 10^{-34}\text{J s})(1.0 \times 10^{8}\text{Hz}) = 6.626 \times 10^{-26}\text{J}
\]

\[
    k T_b = (1.38 \times 10^{-23}\text{J K}^{-1})(4.247 \times 10^{7}\text{K}) = 5.861 \times 10^{-16}\text{J}
\]

Since $h \nu \ll k T_b$, the emission is in the \textbf{Rayleigh-Jeans} regime (long-wavelength limit of the blackbody curve).

\bigskip

\subsection{Effect of Compact Emitting Region on Brightness Temperature}

If the actual emitting region is more compact, the angular diameter $\theta$ is smaller, leading to a smaller solid angle $\Omega$.

Since:

\[
    T_b \propto \dfrac{1}{\Omega}
\]

A smaller $\Omega$ results in a higher brightness temperature $T_b$.

\bigskip

\subsection{Frequency at Which the Radiation Peaks (Wien's Law)}

Use Wien's Law for frequency:

\[
    \nu_{\text{max}} = \dfrac{k T}{h} \times x
\]

Where:
\begin{itemize}
    \item $k$ is Boltzmann's constant
    \item $h$ is Planck's constant
    \item $x$ is a constant approximately equal to 2.8214
\end{itemize}

Calculate $\nu_{\text{max}}$:

\[
    \nu_{\text{max}} = \dfrac{(1.38 \times 10^{-23}\text{J K}^{-1})(4.247 \times 10^{7}\text{K})}{6.626 \times 10^{-34}\text{J s}} \times 2.8214 = 2.497 \times 10^{18}\text{Hz}
\]

\newpage

\section{Problem 5}


\subsection{Optically Thin Cloud: Brightness as a Function of b}

For an optically thin cloud, the observed brightness \( I_\nu(b) \) at Earth is the integrated emission along the line of sight at a distance \( b \) from the cloud center. The emission coefficient \( j_\nu \) (power emitted per unit volume per unit solid angle per unit frequency) relates to \( P(\nu) \) as:

\[
    j_\nu = \frac{P(\nu)}{4\pi}
\]

The path length through the cloud at impact parameter \( b \) is:

\[
    L(b) = 2 \sqrt{R^2 - b^2} \quad \text{for } b \leq R
\]

Thus, the brightness \( I_\nu(b) \) is:

\[
    I_\nu(b) = \int_{-\infty}^{+\infty} j_\nu(s) \, ds = \frac{P(\nu)}{4\pi} \times 2 \sqrt{R^2 - b^2} = \frac{P(\nu)}{2\pi} \sqrt{R^2 - b^2}
\]

\subsection{Optically Thin Cloud: Effective Temperature}

\( L = (4/3) \pi R^3 P_\nu \), where \( P = \int P_\nu \, d\nu \).

\[
    L = 4 \pi R^2 \sigma T_{\text{eff}}^4,
\]

\[
    T_{\text{eff}} = \left( \frac{P R}{3 \sigma} \right)^{1/4}.
\]

\subsection{Optically Thin Cloud: Flux \( F_\nu \) Measured at Earth}

The total luminosity \( L_\nu \) emitted by the cloud at frequency \( \nu \) is:
\[
    L_\nu = P(\nu) \times V = P(\nu) \times \frac{4}{3} \pi R^3
\]
The flux \( F_\nu \) measured at Earth is:
\[
    F_\nu = \frac{L_\nu}{4\pi d^2} = \frac{P(\nu) \times \frac{4}{3} \pi R^3}{4\pi d^2} = \frac{P(\nu) R^3}{3 d^2}
\]

\subsection{Optically Thin Cloud: Comparison of Brightness Temperatures}

Since \( T_B(b) \propto \sqrt{R^2 - b^2} \) and \( P(\nu) \) is finite, the measured brightness temperatures \( T_B(b) \) are much less than the actual cloud temperature \( T \):
\[
    T_B(b) \ll T
\]
This is because the cloud is optically thin and does not emit as a blackbody along the line of sight.

\subsection{Optically Thick Cloud}

\subsubsection{Brightness as a Function of b}

For an optically thick cloud, the specific intensity \( I_\nu(b) \) at any impact parameter \( b \leq R \) is equal to the blackbody intensity:
\[
    I_\nu(b) = B_\nu(T)
\]
since the cloud is opaque and emits as a blackbody at temperature \( T \).

\subsubsection{Effective Temperature}

In this case, the brightness temperature is equal to the actual temperature:
\[
    T_B(b) = T
\]

\subsubsection{Flux \( F_\nu \) Measured at Earth}

The flux measured at Earth is calculated from the projected area of the cloud and the blackbody intensity:
\[
    F_\nu = \frac{\text{Total Power Emitted Towards Earth}}{\text{Area at Distance } d} = \frac{I_\nu(b) \times \text{Projected Area}}{d^2} = \frac{B_\nu(T) \times \pi R^2}{d^2}
\]

\subsubsection{Comparison of Brightness Temperatures}

For an optically thick cloud, the measured brightness temperatures are equal to the cloud's temperature:
\[
    T_B(b) = T
\]
This is because the cloud emits as a blackbody along every line of sight within its angular size.

\newpage

\bibliographystyle{plain}
\bibliography{references}
\nocite{El-Deeb_PEU-438_Assignments}

\end{document}